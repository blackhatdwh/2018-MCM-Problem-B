%%
%% This is file `mcmthesis-demo.tex',
%% generated with the docstrip utility.
%%
%% The original source files were:
%%
%% mcmthesis.dtx  (with options: `demo')
%% 
%% -----------------------------------
%% 
%% This is a generated file.
%% 
%% Copyright (C)
%%     2010 -- 2015 by Zhaoli Wang
%%     2014 -- 2016 by Liam Huang
%% 
%% This work may be distributed and/or modified under the
%% conditions of the LaTeX Project Public License, either version 1.3
%% of this license or (at your option) any later version.
%% The latest version of this license is in
%%   http://www.latex-project.org/lppl.txt
%% and version 1.3 or later is part of all distributions of LaTeX
%% version 2005/12/01 or later.
%% 
%% This work has the LPPL maintenance status `maintained'.
%% 
%% The Current Maintainer of this work is Liam Huang.
%% 
\documentclass{mcmthesis}
\mcmsetup{CTeX = false,
        tcn = 87076, problem = B,
        sheet = true, titleinsheet = true, keywordsinsheet = true,
        titlepage = false, abstract = true}
\usepackage{palatino}
\usepackage{lipsum}
\title{2068: A Linguistic Odyssey}
\begin{document}
\begin{abstract}
\begin{keywords}
keyword1; keyword2
\end{keywords}
\end{abstract}
\maketitle
\tableofcontents
\newpage
\section{Introduction}
\subsection{Background}
Since the first time an ape accidentally vibrates his vocal cords in a strange way and generated a sound different from the wild roar, a splendid voyage of human language begins. An enormous amount of languages had been created during the long journey of evolution. Some of them are flourishing and others are gradually extinct. Up until now, roughly 6900 languages survived and are still spoken on earth.

But change never stops, especially in this era of globalization. Promotion by governments, education in school, pressure from society, migration among countries, etc. Multiple factors contribute to languages' rise and fall. Moreover, with the help of modern industry and the internet, geographically isolated languages can interact with each other easily, which may lead to more uncertainty. Living in such a rapidly changing world, how can we assess languages' future?

Being hired by the COO of an international company, we are required to propose a model which can predict the distribution of various language speakers over time. We are also expected to run the model to predict the trends in the number of native speakers and total language speakers in the next 50 years. Combining this result with the expectation of global population and migration patterns, prediction of geographic distributions of these languages can also be made.

To help our client achieve tangible benefits, we need to use our model to find out six locations for them to open new international offices, and choose the language for each office. Furthermore, we should take the evolution of communication technologies into consideration and evaluate the necessity of opening six new offices.
\subsection{Assumptions}
\subsection{Notations}
\begin{itemize}
    \item \textbf{Everyone born in a given country will learn the native language of that country as her first language, and learn other official languages if there exists some.} This is because no matter what language her parents speak, official languages must to taught as a primary course in the school she attends. This is provisioned by law in most countries.
\end{itemize}


\section{Model}
\section{Verification}
\section{Conclusion}

\begin{thebibliography}{99}
\bibitem{1} D.~E. KNUTH   The \TeX{}book  the American
Mathematical Society and Addison-Wesley
Publishing Company , 1984-1986.
\bibitem{2}Lamport, Leslie,  \LaTeX{}: `` A Document Preparation System '',
Addison-Wesley Publishing Company, 1986.
\bibitem{3}\url{http://www.latexstudio.net/}
\bibitem{4}\url{http://www.chinatex.org/}
\end{thebibliography}

\begin{appendices}

\section{First appendix}

\lipsum[13]

Here are simulation programmes we used in our model as follow.\\

\textbf{\textcolor[rgb]{0.98,0.00,0.00}{Input matlab source:}}
\lstinputlisting[language=Matlab]{./code/mcmthesis-matlab1.m}

\section{Second appendix}

some more text \textcolor[rgb]{0.98,0.00,0.00}{\textbf{Input C++ source:}}
\lstinputlisting[language=C++]{./code/mcmthesis-sudoku.cpp}

\end{appendices}
\end{document}

%% 
%% This work consists of these files mcmthesis.dtx,
%%                                   figures/ and
%%                                   code/,
%% and the derived files             mcmthesis.cls,
%%                                   mcmthesis-demo.tex,
%%                                   README,
%%                                   LICENSE,
%%                                   mcmthesis.pdf and
%%                                   mcmthesis-demo.pdf.
%%
%% End of file `mcmthesis-demo.tex'.

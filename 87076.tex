%%
%% This is file `mcmthesis-demo.tex',
%% generated with the docstrip utility.
%%
%% The original source files were:
%%
%% mcmthesis.dtx  (with options: `demo')
%% 
%% -----------------------------------
%% 
%% This is a generated file.
%% 
%% Copyright (C)
%%     2010 -- 2015 by Zhaoli Wang
%%     2014 -- 2016 by Liam Huang
%% 
%% This work may be distributed and/or modified under the
%% conditions of the LaTeX Project Public License, either version 1.3
%% of this license or (at your option) any later version.
%% The latest version of this license is in
%%   http://www.latex-project.org/lppl.txt
%% and version 1.3 or later is part of all distributions of LaTeX
%% version 2005/12/01 or later.
%% 
%% This work has the LPPL maintenance status `maintained'.
%% 
%% The Current Maintainer of this work is Liam Huang.
%% 
\documentclass{mcmthesis}
\mcmsetup{CTeX = false,
        tcn = 87076, problem = B,
        sheet = true, titleinsheet = true, keywordsinsheet = true,
        titlepage = false, abstract = true}
\usepackage{palatino}
\usepackage{lipsum}
\usepackage{tabularx}
\usepackage{amsmath}
\usepackage{amssymb}
\usepackage{nicefrac}
\usepackage{url}
\newcommand\bsfrac[2]{%
\scalebox{-1}[1]{\nicefrac{\scalebox{-1}[1]{$#1$}}{\scalebox{-1}[1]{$#2$}}}%
}

\title{2068: A Linguistic Odyssey}
\begin{document}
\begin{abstract}
\begin{keywords}
keyword1; keyword2
\end{keywords}
\end{abstract}
\maketitle
\tableofcontents
\newpage
\section{Introduction}
\subsection{Background}
Since the first time an ape accidentally vibrates his vocal cords in a strange way and generated a sound different from the wild roar, a splendid voyage of human language begins. An enormous amount of languages had been created during the long journey of evolution. Some of them are flourishing and others are gradually extinct. Up until now, roughly 6900 languages survived and are still spoken on earth.

But change never stops, especially in this era of globalization. Promotion by governments, education in school, pressure from society, migration among countries, etc. Multiple factors contribute to languages' rise and fall. Moreover, with the help of modern industry and the internet, geographically isolated languages can interact with each other easily, which may lead to more uncertainty. Living in such a rapidly changing world, how can we assess languages' future?

Being hired by the COO of an international company, we are required to propose a model which can predict the distribution of various language speakers over time. We are also expected to run the model to predict the trends in the number of native speakers and total language speakers in the next 50 years. Combining this result with the expectation of global population and migration patterns, prediction of geographic distributions of these languages can also be made.

To help our client achieve tangible benefits, we need to use our model to find out six locations for them to open new international offices, and choose the language for each office. Furthermore, we should take the evolution of communication technologies into consideration and evaluate the necessity of opening six new offices.
\subsection{Assumptions}\label{ssec:1}
\begin{enumerate}
    \item \textbf{Every country has a native language. Everyone born in that country can master it.} The native language refers to the most commonly used language in a country. Living in an environment where most people around you speak that language, it is reasonable to infer that you can master that language.
    \item \textbf{Some countries have an official promoted language. Educated people in that country can master it.} The official promoted language refers to the most commonly studied foreign language in a country. Since it is taught in most schools in that country, we can assume that people finish their basic education can master it.
    \item \textbf{The birth / death / immigration / emigration rate of every country is a constant.} This is a helpless choice, given that we were unable to find enough historical data to predict future trends.
    \item \textbf{Among all the immigrants in a given country, the distribution of languages they use is the same as the distribution of languages in the world.} Here we assume that immigrants come from all over the world evenly to simplify our model.

\end{enumerate}
\subsection{Notations}

\begin{tabularx}{\textwidth}{l|X}
\hline
  \textbf{Symbol} & \textbf{Meaning} \\
\hline
    $S_l$ & The set of 26 languages mentioned in the problem appendix.\\ \hline
    $S_j$ & The set of countries with over 10 million native speakers of language $j$, where $j \in S_l$.\\ \hline
    $S_c$ & The union of $S_j$, $\forall j \in S_l$.\\ \hline
    $N_t$ & The total population on Earth in year $t$.\\ \hline
    $A_i$ & The \textit{i} th country in $S_c$.\\ \hline
    $N_{it}$ & The population of the \textit{i} th country in $S_c$ in year $t$.\\ \hline
    $B_i$ & The birth rate of the \textit{i} th country in $S_c$.\\ \hline
    $D_i$ & The death rate of the \textit{i} th country in $S_c$.\\ \hline
    $I_i$ & The immigration rate of the \textit{i} th country in $S_c$.\\ \hline
    $E_i$ & The emigration rate of the \textit{i} th country in $S_c$.\\ \hline
    $L_{jt}$ & The proportion of people using language $j$ in year $t$ in the world.\\ \hline
    $L_{ijt}$ & The proportion of people using language $j$ in year $t$ in country $i$.\\ \hline
    $L_{int}$ & The proportion of people using the native language in year $t$ in country $i$.\\ \hline
    $x_i$ & The education index\cite{wiki:edu} of country $i$.\\ \hline
    $y_i$ & The proportion of internet user in country $i$.\\ \hline
    $z_j$ & The international influence index of language $j$. $z_j=T_j F_j/TF$, where $T_j$ is the number of tourists travel to $S_j$, $T$ is the total number of tourists in the world, $F_j$ is the foreign trade amount in $S_j$, and $F$ is the total foreign trade amount in the world.\\ \hline

\end{tabularx}


\section{Model}
\subsection{Model One}
We apply difference equation to form our model for part one of the problem. The evolution and competition of languages are affected by economic, social and environmental factors, which finally forms a complicated and unpredictable system. But by the usage of recurrence relation, things get much simpler. We can take all possible factors into consideration and form a extremely complex recurrence formula, but once we get the initial state of the system, we can let the computer handle everything else.

We classify languages spoken in a given country into three categories, the native language, the official promoted language, and other languages. They were introduced in section \ref{ssec:1}, and they change in three different patterns.

%Formula 1
First, let's review the recurrence formula of the native language usage rate (\textbf{NLUR}) of a country, which is shown in Formula \ref{eq:1}.
\begin{equation} \label{eq:1}
    %L_{int} = \frac{L_{int'}N_{it'}(1+B_{it'}-D_{it'}-E_{it'})+\nicefrac{N_{it'}I_i{L_{nt'}}^\alpha}{L_{int'}}+N_{it'}(1-L_{nt'}){L_{int'}}^\eta}{N_{it'}(1+B_{it'}-D_{it'}+I_{it'}-E_{it'})}
    L_{int} = \frac{L_{int'}N_{it'}(1+B_{it'}-D_{it'}-E_{it'})+N_{it'}I_i{L_{nt'}}^\alpha/L_{int'}+N_{it'}(1-L_{nt'}){L_{int'}}^\eta}{N_{it'}(1+B_{it'}-D_{it'}+I_{it'}-E_{it'})}
\end{equation}
\begin{equation*}
    t'=t-1
\end{equation*}


Since the NLUR is obtained by native speaker population divided by the total population, if we are to predict country $i$'s NLUR in year $t$, with $i$'s NLUR and population in year $t-1$ available, then we will need to get the change of total population and the native speaker population in $i$ during the year. The total population is affected by birth/death/immigration/emigration rates, which is described in the denominator, given that birth/immigration increase its population, while death/emigration decrease its population. The population of native speaker will decrease as native speakers emigrate or die, but increase as children born and immigrants arrive. Newborns will definitely learn the native language, while there is a probability for immigrants to learn the local language, as shown in the second term of the numerator: $N_{it'}I_i$ is the number of immigrants in year $t'$, and \nicefrac{${L_{nt'}}^\alpha$}{$L_{int'}$} is the probability, denoted as $p_1$ in this article.

$p_1$ is determined by two factors, the popularity of the language around the world and the social pressure in the country. Given the fact that people tend to learn popular languages rather than minority languages, it is reasonable to establish a positive correlation between the popularity and $p_1$. This is reflected in the numerator of $p_1$, where $L_{nt'}$ is the proportion of people using this language around the world in year $t'$, and $\alpha$ is a constant and set to be ??? here. $L_{nt'}$ is obtained by formula below:
\begin{equation*}
    L_{nt'}=\sum L_{int'}, \forall i \in S_c
\end{equation*}
The social pressure works like this: when too many foreigners immigrate to a country, local people will feel discomfort, since so many people who speak languages they do not understand suddenly appear beside them, taking away their jobs and bringing instability to society. Local people will protest and ask the government to promote the social integration of immigrants. Then the government will take multiple actions, including providing free language courses to immigrants. So we can assume that the higher social pressure, the more possible for immigrants to learn the native languages. The social pressure is measured by NLUR in the country in the previous year, the less NLUR (which means more foreigners are immigrating to the country), the more social pressure. Since $p_1$ and social pressure are negatively correlated, while social pressure and NLUR are possitively correlated, we treat NLUR as the denominator of the $p_1$.

Next, those who cannot speak the native language in the previous year may learn to speak it in this year with probability $p_2$, due to some personal consideration. These personal factors are hard to measure, so we simply assume that $p_2$ equals ${L_{int'}}^\eta$, in which $\eta$ is a constant and set to be ???. We make this assumption because we think that if the native language is spoken by most people in the country, those who cannot speak it will find it difficult to live in this country, and have more reason to learn this language. So in the third term of Formula \ref{eq:1}'s numerator, we multiply $N_{it'}$ and $(1-L_{nt'})$ to obtain the amount of people who cannot speak the native language, then multiply it by $p_2$ to obtain the amount of new learners in this year.

%Formula 2
With the ability to predict the NLUR, let's continue to review the recurrence formula of the official promoted language usage rate (\textbf{OLUR}), as shown in Formula \ref{eq:2}.
\begin{equation} \label{eq:2}
    L_{iot}=\frac{L_{iot'}N_{it'}(1-D_{it'}-E_{it'})+N_{i t-18}B_{it-18}{x_i}^\gamma {y_i}^\beta}{N_{it'}(1+B_{it'}-D_{it'}+I_{it'}-E_{it'})}
\end{equation}
\begin{equation*}
    t'=t-1
\end{equation*}

Recall we have assumed that only educated people can master the official promoted language. But educated people is a fairly broad concept which is hard to define (and hard to collect corresponding data either), so here we use the latest \textbf{Education Index} published by the United Nations as a rough measure of the level of education in a country. The index is calculated from the \textit{Mean years of schooling Index} and \textit{Expected years of schooling Index}. The higher the index, the higher the level of education. Thus, the higher the index, the more possible for grown ups to master the official promoted language.

Also, easier internet access will result in easier learning, since people can learn foreign languages on the internet, in addition to the classroom. We measure the convenience of internet access in a country by the proportion of internet users in the total population of that country. As before, the higher the proportion, the more possible for grown ups to master the official promoted language. We ignore people under 18 years old here because most of them tend to play online games rather than learn languages with their computer.

Formula \ref{eq:2} is basically the same as Formula \ref{eq:1}, only has little difference in numerator. Here we ignore newborns in the numerator, for they are too young to learn a second language. But we take those who have just reached their adulthood into consideration, since at this age, they have probably completed their high school education and have the ability to master the official promoted language. To obtain the number of people who has just reach their adulthood, we simply calculate the amount of newborns 18 years ago. Then we multiply it by ${x_i}^\gamma {y_i}^\beta$ to form the amount of people who have just became adult and can master the official promoted language. Here $\gamma$ and $\beta$ are two constants, $\gamma$ is set to be ???, and $\beta$ is set to be ???.

%Formula 3
At last, we are going to review the recurrence formula of language $j$, which is not the native or official promoted languages in a country $i$. It is shown in Formula \ref{eq:3}.

\begin{equation} \label{eq:3}
    L_{ijt}=\frac{L_{ijt'}N_{it'}(1-D_{it'}-E_{it'})+L_{jt'}N_{it'}I_{ijt'}+{L_{jt'}}^\alpha {z_j}^\theta {x_i}^\gamma {y_i}^\beta N_{it'} (1-I_{ijt'})}{N_{it'}(1+B_{it'}-D_{it'}+I_{it'}-E_{it'})}
\end{equation}
\begin{equation*}
    t'=t-1
\end{equation*}

The denominator of Formula \ref{eq:3} is the same as the previous two formulas, they all represent the population in country $i$ in year $t$. The numerator contains three terms, the first term represents the remaining population of language $j$ speaker in country $i$, after removing emigration and death. The second term is obtained according our fourth assumption, that among all the people immigrate to coutry $i$ in year $t-1$, $L_{jt'}$ of them can speak language $j$. So the second term calculates the population of language $j$ speaker among all the people immigrates to country $i$ in year $t'$.

Those natives of country $i$ who do not speak language $j$ also have some possibilities to learn it. The probability $p_3$ is determined by the level of education ${x_i}^\gamma$ in country $i$, the level of internet usage ${y_i}^\beta$ in country $i$, the international influence index ${z_j}^\theta$ of language $j$, and the popularity ${L_{jt'}}^\alpha$ of language $j$ in the world. The third term in the numerator multiply all of these factors by the amount of non-language $j$ speakers in all local people to obtain the population of natives who learned language $j$ during year $t$.

Now that we have established the recurrence relation of the usage rate of all languages spoken in a given country, we only need the initial state to run the model.


\subsection{Model Two}

\section{Verification}
\section{Conclusion}
Predicting the future of languages is an intricate problem which should take multiple factors into consideration. In our first model, we used three formulas to established the recurrence relation of \textit{Native Language Usage Rate ($L_{int}$)}, \textit{Official Promoted Language Usage Rate ($L_{iot}$)} and \textit{Other Language Usage Rate ($L_{ijt}$)} correspondingly. Then we assessed the current state of language distribution based on massive data we collected, and run the model with the current state as the initial state. The result shows the demographic and geographic distribution of 26 mainstream languages in the next 50 years.

In our second model, we first selected 20 cities as candidates, and used AHP to rate them. Among all factors used in the rating system, we assigned language with the heaviest weight to emphasize the impact of language on choosing addresses.
\subsection{Strength}
\begin{itemize}
    \item Our model has good performance in terms of robustness, proved by the result of sensitivity analysis.
    \item Our model's prediction is about the same as those prevailing in the world today, namely English will become the most popular language in the world.
    \item Our model clearly reflects the relationship between language change and population growth. The latter one, from our perspectives, is one of the most significant factors that affact the future of the world.
    \item Our model take many factors into consideration, including all the factors mentioned in the problem text, and some of our own invention, such as the international influence of languages, and the convenience of internet access.
    \item We have searched a large amount of data from trusted sources to form our assessment of the current linguistic distribution in the world. With a credible initial state, our prediction can be considered as trustworthy.
\end{itemize}
\subsection{Weakness}
\begin{itemize}
    \item We didn't take most political hot issues into consideration, like British Brexit, European migrant crisis, Chinese peaceful rise, and nuclear crisis in North Korea, etc. These issues may lead us to a totally different future, but due to limited time, they are beyond our reach in the current model.
    \item We assume that one can never forget the language she learn to simplify our model, but this is unlikely to happen in the real world.
    \item We assume that a country's native language and official promoted language never change. But such thing may happen, albeit with a very low probability.
    \item Although we have found large amount of data, most of them contain only data in one year. Without enough historical data, we were unable to predict their trend in the future, so we can only regard them as constants. This flaw led our model to maintain good performance only in the near future, with much less accuracy in the far future.
\end{itemize}
\section{A Memo to Chief Operating Officer}
\noindent\textbf{To:} Chief Operating Officer

\noindent\textbf{From:} HAL9000

\noindent\textbf{Date:} February 12, 2018

\noindent\textbf{Subject:} The Future of Languages and Chance to Seize

\noindent There has been some progress in the research you have commissioned to us. According to our model, 



\bibliography{87076}
\bibliographystyle{ieeetr}




\begin{appendices}

\section{First appendix}


Here are simulation programmes we used in our model as follow.\\

\textbf{\textcolor[rgb]{0.98,0.00,0.00}{Input matlab source:}}
\lstinputlisting[language=Matlab]{./code/mcmthesis-matlab1.m}

\section{Second appendix}

some more text \textcolor[rgb]{0.98,0.00,0.00}{\textbf{Input C++ source:}}
\lstinputlisting[language=C++]{./code/mcmthesis-sudoku.cpp}

\end{appendices}
\end{document}

%% 
%% This work consists of these files mcmthesis.dtx,
%%                                   figures/ and
%%                                   code/,
%% and the derived files             mcmthesis.cls,
%%                                   mcmthesis-demo.tex,
%%                                   README,
%%                                   LICENSE,
%%                                   mcmthesis.pdf and
%%                                   mcmthesis-demo.pdf.
%%
%% End of file `mcmthesis-demo.tex'.
